\section{Assesserings Proses}
    Verwys na die eksamen regulasies in die jaarboek van die Fakulteit
    Ingenieurswese, Bou-Omgewing en Inligtingtegnologie
    (\url{http://www.up.ac.za/af/yearbooks/faculties}).

    \noindent
    Om die module deur te kom moet die student:
    \begin{itemize}
        \item `n Finale punt van 50\% behaal; {\bf en}
        \item `n Sub-minimum van 40\% vir die semester punt; {\bf en}
        \item `n Sub-minimum van 40\% vir die finale eksamen
    \end{itemize}

    \subsection{Berekening van die Finale Punt}
        Die finale punt word as volg bereken:
        \begin{itemize}
            \item Semester punt: 50\%
            \item Finale eksamen: 50\% (3-uur eksamen), geslote boek
              agter `n rekenaar
        \end{itemize}

    \subsection{Berekening van die Semesterpunt}
        Besonderhede rakende die berekening van die semesterpunt word in die
        volgende tabel gelys:
        \begin{table}[!h]
            \begin{center}
             \begin{tabular}{|p{10cm}|c|l|l|}
               \hline
               {\bf Evaluasie Metode} & {\bf Aantal} & {\bf Totaal} \\
               \hline
               Semestertoetse (geslote boek agter `n rekenaar)
               & 2 & {\bf 80\%} \\ \hline
               Individuele projek
               & 1 & {\bf 10\%} \\ \hline
               Groep projek
               & 1 & {\bf 10\%} \\
               \hline
               \multicolumn{2}{|l|}{{\bf Total}} & {\bf 100\%} \\
               \hline
             \end{tabular}
             \caption{Semesterpunt berekening}
            \end{center}
        \end{table}

    \subsubsection{Semestertoetse}
        Twee semester toetse sal geskryf word gedurende die semester, in die
        week van 10 to 17 Maart 2018 en 5 tot 12 Mei 2018. Semester
        toetse is 90 minute lank. Die sillabus wat in die toets gedek word sal
        die week voor toetsweek bespreek word. Altwee toetse sal geslote boek
        wees. Die toetse sal in die rekenaar labs geskryf word waar die
        studente toegang sal h\^{e} tot Python en LibreOffice.

        Bykomende semester toets instruksies sal die voorafgaande week tydens
        die lesings gegee word en op \textit{ClickUP} gesit word.

        Die oplossingsblad van die geskeduleerde toetse sal in elektroniese
        formaat op \textit{ClickUP} beskibaar wees na die onderskeie toetse.
        Die student is welkom om gebruik te maak van die besprekingsbord op
        \textit{ClickUP} vir enige verdere vrae oor die toetse.

    \subsubsection{Weeklikse Opdragte} \label{sec:tutoriaal}
        Weeklikse opdragte sal op \textit{ClickUP} gelaai word vir inhandiging
        voor die gespesifiseerde datum. Weeklikse opdragte sal gebruik word vir
        verdere ontwikkeling van die studente se begrip en kennis oor die
        lesingsonderwerpe. Weeklikse opdragte sal nie in die berekening van die
        semesterpunt tel nie, maar sal die student in `n goeie plek stel vir
        die voorbereiding van toetse en eksamens.

        Weekliks sal hierdie opdragte elektronies gemerk word en terugvoering
        aan die studente verskaf word. Neem kennis dat die elektroniese
        graderingstelsel slegs 0\% of 100\% op elke vraag kan verwerf. Dit is
        ook uiters belangrik dat die inhandigingsinstruksie op le\^ernaam,
        objeknaam, funksienaam, ens. noukeurig gevolg word anders sal die
        graderingstelsel 0\% verwerf.

    \subsubsection{Individuele Projek}
        Gedetailleerde inligting oor die individuele projek sal later op
        \textit{ClickUP} beksikbaar gemaak word.

        \textbf{Let Op:} Soos die naam aandui, moet die individuele projek op
        `n individu\"ele basis gedoen word. Alle elektroniese opdragte wat
        ingedien word sal vir plagiaat nagegaan word. Enige plagiaat
        oortredings sal nie toegelaat word nie.  Verwys na die Department\"ele
        Studiegids (sien afdeling \ref{sec:department}) vir verdere inligting
        oor plagiaat.

    \subsubsection{Groep Projek}
        Die groep semester projek is `n geleentheid vir die studente om saam te
        werk as `n groep om `n gemeenskaplike probleem op te los.

        `n ``Groep Projekgids'', met besonderhede en vereistes rakende die
        projek, sal later op \textit{ClickUP} beksikbaar gemaak word.

        \textbf{Let Op:} Alle elektroniese opdragte wat ingedien word sal vir
        plagiaat nagegaan word. Enige plagiaat oortreedings sal nie toegelaat
        word nie.  Verwys na die Department\"ele Studiegids (sien afdeling
        \ref{sec:department}) verdere inligting oor plagiaat.

    \subsection{App\`{e}lle en navrae oor punte}
        Die punte toegeken vir die tutoriaalopdragte en semestertoetse sal
        beskikbaar gemaak word op \textit{ClickUP}. Indien die student enige
        navrae het oor die toegekende punt, moet die student asseblief binne 14
        dae vanaf die punte ontvang is die prosedure onder volg.  Na die
        verloop van 14 dae sal geen veranderinge aangebring word nie. Sien die
        departementele studiegids in Afdeling \ref{sec:department} vir verdere
        inligting.

    \subsubsection{App\`{e}l proses vir semestertoetse}
        \begin{enumerate}
            \item Laai die memorandum en navraagvorm van \textit{ClickUP} af.
            \item Gaan noukeurig deur jou toetsvraestel en memorandum.
            \item Vul die navraagvorm uit en dui aan wat die navraag behels.
            \item Handig jou vraestel aan die dosent teen die einde van die
                lesing binne die 14 dae periode.
        \end{enumerate}

        \textbf{Geen vraestel sal aanvaar word sonder `n aangehegde navraagvorm
        nie.}  Die dosent sal wag tot al die vraestelle ontvang is en tot die
        14 dae verby is voor die hersieningsproses aangepak word.
