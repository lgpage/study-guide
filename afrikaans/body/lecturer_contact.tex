\section{Dosente en Konsultasie Ure}
    \subsection{Dosente}
        \begin{table}[!h]
            \begin{center}
             \begin{tabular}{|l|l|c|l|}
                 \hline
                 {\bf Dosent} & {\bf Kantoor} & {\bf Telefoon No.} & {\bf Epos} \\
                 \hline
                 Mnr Logan Page &
                 Ing III, 6--93 &
                 (012) 420 6891 &
                 \href{mailto:logan.page@up.ac.za}{logan.page@up.ac.za} \\
                 %
                 Mnr Stephan Roux &
                 Eng III 6--91 &
                 (012) 420-2935 &
                 \href{mailto:stephan.roux@up.ac.za}{stephan.roux@up.ac.za} \\
                 \hline
             \end{tabular}
            \end{center}
        \end{table}

        Gebruik asb. die sleutelwoord \textbf{MPR213} in die onderwerp opskrif
        tydens e-pos korrespondensie.

    \subsection{Onderwysassistente}
        Die kontakbesonderhede en spreekure vir die onderwysassistente sal
        beskikbaar gestel word op \textit{ClickUP} sodra die inligting
        beskikbaar is.

    \subsection{Konsultasie Ure}
        Konsultasie ure vir die dosente en onderwysassistente sal aan die begin
        van die semester aangekondig word.  Hierdie ure sal aangedui word by
        hulle kantoordeure en op \textit{ClickUP.}  Hierdie ure dui aan wanneer
        die dosente en onderwysassistente beskikbaar is op kampus vir
        konsultasie.  Studente word versoek om `n afspraak te maak om te
        verseker dat die dosent en/of onderwysassistente wel in hulle kantore
        is.

        Studente is welkom om spesiale re\"{e}lings te tref om dosente buite
        hierdie konsultasie ure te sien indien nodig.  Dit is belangrik om nie
        te wag tot net voor die eksamen om probleme op te klaar nie.  Daar sal
        geen buitengewone spreekure gedurende die toets en eksamen tyd wees
        nie.

        Studente is ook welkom om gebruik te maak van die besprekingsbord
        op \textit{ClickUP}.  Dit sal dosente, onderwysassistente en ander
        medestudente in staat stel om, gedurende hul vrye tyd, enige vrae te
        beantwoord rakende programmeringskonsepte, oefenprobleme, ens.
