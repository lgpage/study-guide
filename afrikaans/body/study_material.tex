\section{Studie Materiaal en Sagteware}
    \subsection{Voorgeskrewe Handboek}
        Daar is geen voorgeskrewe handboek vir hierdie kursus nie. Die volgende
        \textbf{studienotas} is by UP ontwikkel en sal beskikbaar wees op
        \textit{ClickUP}:

        ``\underline{Introduction to programming for engineers using Python}''
        deur Logan G. Page, Daniel N. Wilke, en Schalk Kok.

    \subsection{Komplement\^{e}re Bronne}
        Die volgende aanvullende handboek en bron van verwysingsnotas sal ook
        beskikbaar gestel word op \textit{ClickUP}:

        ``\underline{Think Python 2}'' deur Allen Downey

        ``\underline{Python for Computational Science and Engineering}'' deur
        Hans Fangohr

    \subsection{Lesingnotas}
        Let op dat ons die onderskeiding make tussen die \textbf{studienotas}
        wat hierbo bespreek is en die \textbf{lesingnotas} wat hier bespreek
        is. Hierdie lesingnotas sal ook deel van die sillabus wees en is
        beskibaar op \textit{GitHub}:

        \url{https://github.com/mpr213/lecture-notes/releases}

        Hierdie lesingnotas dek nie \'{a}l die werk wat bespreek word gedurende
        klas nie en studente moet asseblief deur die studienotas werk en hulle
        eie aanvulllende notas maak gedurende klastye. Uitgewerkte oplossings
        wat gedurende die lesings in detail gedek word, sal nie later op
        \textit{ClickUP} beskikbaar gemaak word nie.

    \subsection{Vereiste Sagteware}
        In hierdie kursus maak ons gebruik van vrylik-beskikbare sagteware.
        Dit impliseer dat studente die sagteware kan gebruik en kopie\"er
        sonder enige wettige beperkings.

        Die programmeringspakket wat gebruik word in hierdie kursus is
        \textit{Python}.  \textit{Python} is `n ho\"{e}-vlak programmeringstaal
        wat gebruik word in baie dissiplines reg oor die w\^{e}reld.
        \textit{Python} is `n goed geskikte programmeringstaal vir ingenieurs
        omdat dit maklik is om te leer.  \textit{Python} is ook goed
        gedokumenteer, relatief vinnig, en goed geskik vir numeriese en
        ingenieurs berekeninge. \textit{Python} kan deur \textit{Anaconda}
        geinstalleer word, wat afgelaai kan word by
        \url{https://www.anaconda.com/downloads}

        In hierdie kursus word \textit{LibreOffice} gebruik vir sigblaaie
        (spreadsheets). \textit{LibreOffice} is grootendeels soortgelyk aan
        Microsoft Excel, met die uitsondering dat dit afgelaai en gebruik kan
        word sonder wettige beperkings. \textit{LibreOffice} kan afgelaai word
        by \url{http://www.libreoffice.org/download/libreoffice-fresh/}

        Seen \textit{ClickUP} vir additionele informasie,
        installeeringinstruksies en ontfouting.
