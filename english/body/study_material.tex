\section{Study Material and Software}
    \subsection{Prescribed Textbook}
        There is no prescribed textbook for this course, however the following
        \textbf{study notes} have been developed at UP and will be made
        available on {\it ClickUP}:

        ``\underline{Introduction to programming for engineers using Python}''
        by Logan G. Page, Daniel N. Wilke, and Schalk Kok.

    \subsection{Complementary Sources}
        The following complementary textbook and source of reference notes will
        also be made available on {\it ClickUP}:

        ``\underline{Think Python 2}'' by Allen Downey

        ``\underline{Python for Computational Science and Engineering}'' by
        Hans Fangohr

    \subsection{Lecture Notes}
        Take note, we make the distinction here between \textbf{study notes}
        discussed above and \textbf{lecture notes} discussed here. These
        lecture notes will also form part of the syllabus and are available on
        \textit{GitHub}:

        \url{https://github.com/mpr213/lecture-notes/releases}

        These lecture notes do not cover all the work discussed in class and
        students should work through the study notes and take down their own
        supplementary notes during lectures. Problem solutions covered in
        detail during the lectures will not be made available again at a later
        stage.

    \subsection{Required Software}
        In this course, we make use of open source software. This implies that
        students can freely copy and use the software legally without any
        restrictions.

        The programming package used in this course is \textit{Python}.
        \textit{Python} is high level programming language used in many
        disciplines around the world. \textit{Python} is a well suited
        programming language for engineers as it is easy to learn, well
        supported and documented, relatively fast, and well suited for
        numerical and scientific computing. \textit{Python} can be installed
        via \textit{Anaconda}, which can be downloaded from
        \url{https://www.anaconda.com/downloads}.

        \textit{LibreOffice} will be used for spreadsheets in this course. It
        is in many ways very similar to Microsoft Excel with the exception that
        it can be downloaded and used without any restrictions.
        \textit{LibreOffice} can be downloaded from
        \url{http://www.libreoffice.org/download/libreoffice-fresh/}

        See \textit{ClickUP} for additional information, installation
        instructions and troubleshooting.
