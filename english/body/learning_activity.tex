\section{Learning Activities}
    \subsection{Contact Time and Learning Hours}
        \begin{minipage}{0.4\linewidth}
            Number of lectures a week: \\
            Tutorial sessions a week:
        \end{minipage}
        \begin{minipage}{0.4\linewidth}
            4 lectures, 50 minutes per lecture \\
            1 session, 100 minutes per session
        \end{minipage}

        This module carries a weight of 16 credits, indicating that a
        student should spend an average of 160 hours to master the
        required skills (including time spent preparing for tests and
        examinations). This means that you should devote an average of
        11 hours of study time per week to this module. The scheduled
        contact time is approximately 6 hours per week, which
        means that another 5 hours per week of own study time
        should be devoted to the module.

    \subsection{Lectures}
        \begin{table}[!h]
            \begin{center}
             \begin{tabular}{|l|c|p{2.5cm}|p{2.9cm}|p{2.1cm}|p{2.6cm}|p{0.6cm}|}
                 \hline
                 {\bf \#} & {\bf Time} & {\bf Mon.} & {\bf Tues.} & {\bf Wed.} &
                 {\bf Thurs.} & {\bf Fri.} \\
                 \hline
                 1  & 07:30--08:20 &  & [E1]Centenary 4 & [A]Ing III--6 &  & \\ \hline
                 2  & 08:30--09:20 &  & [E1]Centenary 4 & [A]Ing III--6 &  & \\ \hline
                 3  & 09:30--10:20 &  &  &  &  & \\ \hline
                 4  & 10:30--11:20 &  &  &  &  & \\ \hline
                 5  & 11:30--12:20 &  &  &  & [E2]Thuto 1--1 & \\ \hline
                 6  & 12:30--13:20 &  &  &  & [E2]Thuto 1--1 & \\ \hline
                 7  & 13:30--14:20 & [E1]Eng III--7 &  &  &  & \\ \hline
                 8  & 14:30--15:20 & [E1]Eng III--7 &  &  &  & \\ \hline
                 9  & 15:30--16:20 &  & [E2]Thuto 1--2 &  &  & \\ \hline
                 10 & 16:30--17:20 &  & [E2]Thuto 1--2 [A]Ing III--6 &  &  & \\ \hline
                 11 & 17:30--18:20 &  & [A]Ing III--6 &  &  & \\
                 \hline
             \end{tabular}
             \caption{Lecture Timetable}
            \label{tab:lectures}
            \end{center}
        \end{table}

        Mr Roux will present the \textbf{Afrikaans} lectures and Mr Page the
        \textbf{English} lectures as shown in Table \ref{tab:lectures}. The
        lectures are split as per the timetable above. Please refer to your
        individual timetable as to which English lecture group you should
        attend. Also refer to {\it ClickUP} for any timetable updates.

        Lecture attendance and participation in discussions is compulsory.
        Since the contents of each lecture follows on those of previous
        lectures, it is in the student's own interest to study the material
        covered on a regular basis and not to miss a lecture. However, should a
        student not be able to attend a certain lecture for whatever reason,
        the onus is on him / her to obtain the study material and catch up on
        the work. No individual lectures will be presented. Previous material
        will rarely be repeated in a following lecture.

        Students are expected to prepare for lectures. Since a large volume of
        work needs to be covered, it is not possible to cover every aspect in
        the finest detail in the lectures themselves. Students should therefore
        read the textbooks thoroughly and already know beforehand what the next
        lecture is about in order to identify anything that is unclear. The
        lecturer may also assign some sections of the study notes and lecture
        notes / handouts for self-study. These sections will be part of the
        syllabus, but will not be discussed in the class.

    \subsection{Tutorials}
        \begin{table}[!h]
            \begin{center}
            \begin{tabular}{|l|l|l|l|}
                \hline
                {\bf Discipline} & {\bf Day} & {\bf Time} & {\bf Venue} \\
                \hline
                S3          & Mon.   & 11:30-13:20 & NWII Lab 1,2,3,4 \\
                c3 s3 C2    & Mon.   & 15:30-17:20 & NWII Lab 2,3,4 \\
                M2          & Tues.  & 07:30-09:20 & NWII Lab 2,3,4 \\
                n3 p3 N2 P2 & Wed.   & 11:30-13:20 & NWII Lab 1,2,3,4 \\
                b3 B2       & Thurs. & 15:30-17:20 & NWII Lab 1,2,3,4 \\
                m3 M2       & Fri.   & 13:30-15:20 & NWII Lab 1,2 \\
                \hline
            \end{tabular}
            \caption{Tutorial Timetable}
            \label{tab:tutorials}
            \end{center}
        \end{table}

        Tutorial sessions will be used to further develop the students
        understanding and knowledge of the topics covered in lectures
        by solving numerous engineering and mathematical problems with
        the tools introduced in this module. The tutorial sessions are
        tabulated in Table \ref{tab:tutorials}, and allocated
        according to discipline\footnote{Discipline Symbols:
            B -- Industrial and Systems;
            C -- Chemical;
            M -- Mechanical and Aeronautical;
            N -- Materials Science and Metallurgical;
            P -- Mining; and
            S -- Civil. \\
            (uppercase letter -- four year plan;
            lowercase letter -- five year plan)}.

        Each week's tutorial session will be used to review the previous week's
        lecture content and concepts. Selected exercise problems (from the
        study notes) and weekly assignments will be covered during these
        tutorial sessions. The selected chapters, problems and assignments
        covered in each tutorial session will be posted on {\it ClickUP}.

        It is vital that students do these problems on their own, prior to
        attending the tutorial session, and use the tutorial session for
        getting help with problems and / or concepts that the student is
        struggling with. It is important to understand that the tutors are
        there to assist the student to grasp difficult concepts and not to
        solve the problems on their behalf.

        It will be to the benefit of the student to complete these problems, as
        solving these problems will develop the student's thinking and
        programming ability. The tests and exams will test the ability of the
        student to solve new problems of similar complexity. Attendance of the
        tutorial sessions is optional but recommended.
